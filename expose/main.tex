\documentclass[a4paper,12pt]{article}

\usepackage[margin=1in]{geometry} % full-width

% AMS Packages
\usepackage{amsmath}
\usepackage{amsthm}
\usepackage{amssymb}

% Unicode
\usepackage[utf8]{inputenc}
\usepackage{hyperref}
\hypersetup{
    unicode,
    colorlinks,
%   breaklinks,
%   urlcolor=cyan,
    linkcolor=blue,
    pdfauthor={Author One, Author Two, Author Three},
    pdftitle={A simple article template},
    pdfsubject={A simple article template},
    pdfkeywords={article, template, simple},
    pdfproducer={LaTeX},
    pdfcreator={pdflatex}
}

% Vietnamese
%\usepackage{vntex}

% Natbib
\usepackage[sort&compress,numbers,square]{natbib}
\bibliographystyle{plain}

% Theorem, Lemma, etc
% \theoremstyle{plain}
% \newtheorem{theorem}{Theorem}
% \newtheorem{corollary}[theorem]{Corollary}
% \newtheorem{lemma}[theorem]{Lemma}
% \newtheorem{claim}{Claim}[theorem]
% \newtheorem{axiom}[theorem]{Axiom}
% \newtheorem{conjecture}[theorem]{Conjecture}
% \newtheorem{fact}[theorem]{Fact}
% \newtheorem{hypothesis}[theorem]{Hypothesis}
% \newtheorem{assumption}[theorem]{Assumption}
% \newtheorem{proposition}[theorem]{Proposition}
% \newtheorem{criterion}[theorem]{Criterion}
% \theoremstyle{definition}
% \newtheorem{definition}[theorem]{Definition}
% \newtheorem{example}[theorem]{Example}
% \newtheorem{remark}[theorem]{Remark}
% \newtheorem{problem}[theorem]{Problem}
% \newtheorem{principle}[theorem]{Principle}

\usepackage{graphicx, color}
\graphicspath{{figures/}}

%\usepackage[linesnumbered,ruled,vlined,commentsnumbered]{algorithm2e} % use algorithm2e for typesetting algorithms
\usepackage{algorithm, algpseudocode} % use algorithm and algorithmicx for typesetting algorithms
\usepackage{mathrsfs} % for \mathscr command

\usepackage{lipsum}

% Author info
\title{Exposé on Using Large Language Models for Automated Data Extraction from Scientific Literature}
\author{Felix Karg}

\date{\today}
\renewcommand{\abstractname}{\vspace{-\baselineskip}} % clear 'abstract' from abstract


\begin{document}
\maketitle

\begin{abstract}
    It should be noted that everything mentioned in this document is highly
    uncertain, particularly scientific questions, steps and timelines, even if
    not otherwise mentioned.

    % \noindent\textbf{Keywords:} article, template, simple
\end{abstract}

% \tableofcontents

\section{Introduction}
The goal of this work is to create a pipeline that derives accurate answers to
given questions from a provided scientific article. For this, the article
is first provided in some way: either as link to a site to download, as
file-upload or through some other method to be determined later. Next, we do
paragraph extraction similar to \cite{luo_mof_2022} to find paragraphs that
describe synthesis steps. These paragraphs are then provided as context to use
in answering the given questions with a Large Language Model (LLM).
Answers from the extracted data will be returned directly or in a
machine-readable format. From an agglomeration of such extracted data, a
database may be built. This database may in further works be used in works to
make predictions on MOF synthesis procedures.


\section{General Topic and Motivation}
While there are many articles on material synthesis, it is difficult to
automatically extract important information such as reaction time, temperature,
solvent, and additives in a comprehensive database. When trying to automatize
the extraction of relevant data, the problem becomes one of Natural Language
Processing (NLP) and proper semantic understanding of synthesis procedures.

\section{Related Literature}

\begin{itemize}
    \item ChemicalTagger: A tool for semantic text-mining in chemistry \cite{hawizy_chemicaltagger_2011}
        \begin{itemize}
            \item Early demonstration that semantic data extraction works on scientific literature can work
        \end{itemize}
    \item MOF Synthesis Prediction Enabled by Automatic Data Mining and Machine Learning \cite{luo_mof_2022}
        \begin{itemize}
            \item Doing Synthesis prediction based on an automatic data extraction pipeline
        \end{itemize}
    \item Structured information extraction from complex scientific text with fine-tuned large language models \cite{dunn_structured_2022}
        \begin{itemize}
            \item demonstration that LLMs can be very capable of extracting materials chemistry information for representative tasks with high accuracy
        \end{itemize}
\end{itemize}

\section{Scientific Questions}
Benchmarking accuracy of data extraction from scientific literature using state-of-the-art LLMs and other tools: how well do various existing methods do, and (how much) can priming, fine-tuning and prompt engineering improve this.

Specifically, compare the accuracy of:
\begin{itemize}
    \item what the model already knows: here, the article would not be in context
    \item what it can easily extract from the article: when provided in
        context, how well it can answer questions relating the content.
    \item particarly, without much prompt engineering or fine-tuning the model
    \item with prompt engineering: attempt to increase accuracy by finding good prompts for that
    \item after task-based fine-tuning (without context, in-context, and with or without good prompts)
    \begin{itemize}
        \item articles included in fine-tuning
        \item articles not fine-tuned on
        \item experiment with fine-tuning approaches
    \end{itemize}
    \item Models of varying sizes
    \begin{itemize}
        \item the varying sizes of available OPT-models \cite{zhang_opt_2022}
        \item given enough time, try to use distillation \cite{sun_patient_2019} to compress the model parameter size
    \end{itemize}
\end{itemize}

% \section{Objectives}
% Objectives of the work (both mandatory objectives and optional additional objectives)

\section{Intermediate Steps}
% Concrete work steps

\begin{enumerate}
    \item Take exemplary / arbitrary synthesis paper
    \item get paragraph classification to work
    \item figure out how to pass the relevant paragraphs as LLM context
    \item determine accuracy of answering the given questions based on reference database
    \item first without, later with prompt engineering and fine-tuning of the model
\end{enumerate}

\section{Schedule}

\subsection{1-2 Months}
\begin{itemize}
    \item read related papers (those cited before, BERT \cite{devlin_bert_2018}, Attention is All you Need \cite{vaswani_attention_2017}, GPT2 \cite{radford_language_2019}, GPT3 \cite{brown_language_2020}, Chinchilla \cite{hoffmann_training_2022} and more)
    \item follow Andrej Karpathy ML-Course to building a GPT-model yourself
    \item get OPT models of different sizes to run on cluster
    \item build initial pipeline to evaluate OPT models
    \item begin building initial dataset to 
\end{itemize}

\subsection{3-4 Months}
\begin{itemize}
    \item (Keep eyes out for online-mode papers, see if it could be reasonably integrated)
    \item finish building initial dataset to compare accuracy (before fine-tuning)
    \item deep probe models with (reverse) prompts, run prompt engineering experiments
    \item figure out ways to fine-tune, run fine-tuning experiments
\end{itemize}

\subsection{5-6 Months}
\begin{itemize}
    \item Run further experiments
    \item keep eyes out for (new) papers on LLMs
    \item Mostly: Write thesis, re-run experiments (if possible / necessary)
    \item Deploy service, create final database with results
\end{itemize}

\bibliography{references.bib}
 

\end{document}

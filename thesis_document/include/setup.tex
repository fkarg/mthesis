\usepackage{tabularx}
\usepackage{hyperref}
\usepackage{changepage}
\usepackage{subfig}
\usepackage{float}
\usepackage[outputdir=out]{minted}
\usemintedstyle{pastie}
% \usepackage[printonlyused,withpage]{acronym}
% \usepackage[draft]{showlabels}
% \showlabels{cite}
% \showlabels{gls}
% \showlabels{todo}

\usepackage{xparse}
\NewDocumentCommand{\ShowInline}{v}{%
#1%
}
\usepackage{fancyvrb}

\usepackage{csquotes}
\usepackage[acronym,toc,nopostdot]{glossaries}
\setacronymstyle{long-short}
\makeglossaries
../../thesis_document/chapters/acronyms.tex

\interfootnotelinepenalty=10000

\newcommand{\vect}[1]{\boldsymbol{\bm{#1}}} % for package bm

\SelectLanguage{english}
% details on this thesis
\newcommand{\thesisauthor}{Felix Karg}
%\newcommand{\thesistopic}{Name des Themas auf Deutsch}

%\newcommand{\thesisentopic}{Extraction of structural information from powder X-ray diffraction patterns
%using neural network based on randomly generated crystals}
%\newcommand{\thesisentopic}{Analysis of Powder X-Ray Diffractograms
%Using Neural Networks Trained on an Infinite Stream of Synthetic Patterns}
%\newcommand{\thesisentopic}{Analysis of Powder X-Ray Diffractograms
%Using Neural Networks Trained on Continuously Generated Synthetic Patterns}
%\newcommand{\thesisentopic}{Analysis of Powder X-ray Diffractograms Using Neural Networks Based on
%Synthetic Patterns Generated During Training}
%\newcommand{\thesisentopic}{Analysis of Powder X-Ray Diffractograms
%Using Neural Networks Trained on Continuously Generated Synthetic Patterns}
\newcommand{\thesisentopic}{Benchmarking Large~Language~Models for Zero-Shot Automated Information Extraction from Scientific Literature}

\newcommand{\thesisinstitute}{Institute of Theoretical Informatics}
\newcommand{\thesisreviewerone}{T.T.-Prof. Dr. Pascal Friederich}
\newcommand{\thesisreviewertwo}{Prof. Dr. Jan Niehues}
\newcommand{\thesisadvisorone}{Dr. Tobias Schlöder}
\newcommand{\thesistimestart}{2023-03-01} % on titlepage
\newcommand{\longdate}{2nd October 2023} % on signature lines
\newcommand{\thesistimeend}{2023-10-02} % on titlepage
\newcommand{\thesistimehandin}{2023-10-02} % on second page 'preamble'
\newcommand{\thesispagehead}{\thesisentopic} % page heading


% ----------------- referencing ----------------
\newcommand{\secref}[1]{Section~\ref{sec:#1}}
\renewcommand{\subref}[1]{Subsection~\ref{sub:#1}}
\newcommand{\chapref}[1]{Chapter~\ref{chap:#1}}
\renewcommand{\eqref}[1]{Equation~(\ref{eq:#1})}
\newcommand{\figref}[1]{Figure~\ref{fig:#1}}
\newcommand{\tabref}[1]{Table~\ref{tab:#1}}
\newcommand{\coderef}[1]{Code Example~\ref{code:#1}}
\newcommand{\outref}[1]{Output~\ref{out:#1}}
\newcommand{\model}[1]{\hyperref[sub:#1]{\gls*{#1}}}
\newcommand{\draft}[1]{\textcolor{blue}{#1}}
\newcommand{\informal}[1]{\textcolor{red}{\textbf{#1}}\todo{passiv!}}
\newcommand{\passive}[1]{\textcolor{red}{\textbf{#1}}\todo{passiv!}}

\usepackage{xspace}
\newcommand{\ttime}{\texttt{time}\xspace}
\newcommand{\ttemp}{\texttt{temperature}\xspace}
\newcommand{\tadd}{\texttt{additive}\xspace}
\newcommand{\tsolv}{\texttt{solvent}\xspace}
\newcommand{\cid}{\texttt{cid}\xspace}

\newcommand*{\mybox}[2]{\colorbox{#1!10}{\parbox{.98\linewidth}{#2}}}

\newcounter{code}
\setcounter{code}{0}
\newenvironment{codeboxed}[2]{
\begin{minipage}{\linewidth}\begin{center}\textbf{#1}\\\small \mybox{gray}{#2}\\[1ex]\begin{tabular}{|p{\textwidth}|}\hline}{
    \\\hline\end{tabular}\end{center}\end{minipage}}

\newcommand{\code}[3]{\refstepcounter{code}\label{code:#2}\begin{codeboxed}{Code Example \thecode}{#3}
        \inputminted[linenos,fontsize=\small]{Python}{code/#1}\end{codeboxed}}

\newcommand{\mpy}[1]{\mintinline{python}{#1}}

\hypersetup
{
    pdfauthor={\thesisauthor},
    pdftitle={\thesisentopic},
    pdfkeywords={llm,nlp,information extraction,kit,computer science,master,thesis,\thesisauthor}
}

% Describe separation hints here:
\hyphenation
{
    über-nom-me-nen an-ge-ge-be-nen
}

\addbibresource{references.bib}

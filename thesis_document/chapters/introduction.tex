\chapter{Introduction}\label{chap:introduction}

% clear and concise:
% Presentation of the topic and the background
% Motivation of importance
% Research gap, research question
% Structure of the thesis
% Short summary of main results

\gls{ML} models are increasingly used in screening steps for materials discovery and property prediction \cite{saal_machine_2020, luo_mof_2022, choudhary_recent_2022}. 
But the amount of data accessible for such models is limited, particularly for new, experimental or very recent work.

It has been well established that more high-quality data makes models more accurate \cite{hoffmann_empirical_2022}.

Trying to use \glspl{LLM} for \gls{NER} on scientific text synthesizing \glspl{MOF}.

basic problem: access to knowledge buried deep within unknown papers.

% \section{Basics}
% A vast amount of scientific knowledge is scattered across millions of research
% papers. Often, this research is not in standardized machine-readable formats,
% which makes it difficult or impossible to build on prior work using powerful
% tools to extract further knowledge.  \todo{expand on LLMs and limits here?}

% \section{Motivation}
% Take for example the field of synthesizing Metal-Organic Frameworks (MOFs)
% \cite{zhou_introduction_2012}. While numerous detailed descriptions of
% synthesis procedures exist, they are not available in machine-readable formats,
% which prevents effective application of state-of-the-art techniques such as
% automated experimentation \cite{shi_automated_2021} or synthesis prediction
% \cite{luo_mof_2022}. Thus, we intend to create a pipeline for deriving
% machine-readable information on MOF synthesis parameters from given questions
% on provided scientific articles.

% Motivation
% \begin{itemize}
%     \item vast amount of material science knowledge scattered across papers
%     \item often non-machine readable formats
%     \item makes it difficult or impossible to build on (all) prior work
%     \item results in tremendous duplicates and other unnecessary work
%     \item a lot of insights to be gained from such a database, as well as automated experimental design (and possibly execution)
% \end{itemize}

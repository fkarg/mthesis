\chapter{Introduction}\label{chap:introduction}

% clear and concise:
%  Presentation of the topic and the background
Unstructured scientific literature contains a vast amount of materials science knowledge, and is the only place to get it.
However, this places it squarely out of reach for further processing, which makes it difficult or impossible to build on a large part of prior work.
%  Motivation of importance
\gls{ML} models are increasingly used in screening steps for materials discovery and property prediction \cite{saal_machine_2020, luo_mof_2022, choudhary_recent_2022}, among other steps.
% These models continue to get better, but often the biggest bottleneck is a lack of data.
In general, more high-quality data improves the output quality of \gls{ML} models considerably \cite{hoffmann_empirical_2022}.
%  Research gap, research question
Currently, the amount of data accessible to train such models is limited, \todo{@Tobias check} in particular from older or less known work.

Over the last year, \glspl{LLM} rose to public prominence since the launch of \gls{ChatGPT}.
Regardless of public perception, recent improvements across \gls{NLP} benchmarks are undeniable \cite{devlin_bert_2018, openai_gpt4_2023}.
In this work, \glspl{LLM} are used for information extraction of scientific text on synthesisizing \glspl{MOF}.

% Structure of the thesis
\draft{
Structure.
}
% Short summary of main results
\draft{
Results summary.
}

% \section{Motivation}
% Take for example the field of synthesizing Metal-Organic Frameworks (MOFs)
% \cite{zhou_introduction_2012}. While numerous detailed descriptions of
% synthesis procedures exist, they are not available in machine-readable formats,
% which prevents effective application of state-of-the-art techniques such as
% automated experimentation \cite{shi_automated_2021} or synthesis parameter prediction
% \cite{luo_mof_2022}. Thus, we intend to create a pipeline for deriving
% machine-readable information on MOF synthesis parameters from given questions
% on provided scientific articles.

% Motivation
% \begin{itemize}
%     \item vast amount of material science knowledge scattered across papers
%     \item often non-machine readable formats
%     \item makes it difficult or impossible to build on (all) prior work
%     \item results in tremendous duplicates and other unnecessary work
%     \item a lot of insights to be gained from such a database, as well as automated experimental design (and possibly execution)
% \end{itemize}


\section{Scientific Question}\label{sec:question}

The goals of this work are threefold:
\begin{enumerate}
    \item Demonstrate zero-shot automated information extraction from scientific literature using open-access \glspl{LLM}.
    \item Benchmark and compare the accuracy of currently available open-access \glspl{LLM} for the automated information extraction from scientific literature.
    \item Attempt fine-tuning of open-access \glspl{LLM} in order to increase accuracy.
\end{enumerate}

As part of this work, a highly flexible automated pipeline for the extraction of non-machine readable information in \gls{MOF} synthesis will be created.
The approach chosen in this work will be discussed in more detail \chapref{approach}, where the implementation is described in \secref{impl}, the models used in \secref{models}, and \secref{data} describes the source of data.


\newpage
collection of todos
\todo{find a first section for introduction chapter}

explain stuff somewhere
\todo{explain context length somewhere}
\todo{explain zero-shot somewhere}
\todo{explain causal and masked language models somewhere}

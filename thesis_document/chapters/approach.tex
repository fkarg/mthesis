\chapter{Approach}\label{chap:approach}
\todo{approach chapter guideline: what did I do, why did I do it}
Here, we describe some of our approach to using \glspl{LLM}, decisions made, lessons learned, and more.

Specifically, we put little effort in prompt engineering, with the argument that it should have little impact for fine-tuned models, the main part of our work. As it turns out (See \secref{sft} for more details on training failures), this ended up not being the case.

See \subref{list} for a list of the models used, \subref{criteria} for selection criteria for the models, and the respective Subsection for more details on each model, which is als linked in their glossary entry.
\todo{rewrite when the rest becomes clearer}

All source code for this work can be found at \url{https://github.com/fkarg/mthesis}, and a tag will mark the state at the time of submission.

\section{Implementation}\label{sec:impl}
As partially described before in \secref{models}, it became appearent during literature research that it might be valuable to compare different models, and that additional models might become available in the near future.
The \acrlong{transformers} library is a well-established framework providing abstractions to load, manipulate and train any deep learning architecture in a standardized format.
Additionally, all open-access \glspl{LLM} are available directly through the \gls{hf} portal.

Most other libraries used are either straightforward (\texttt{torch, einops, accelerate, bitsandbytes} to get the models to run) or common ecosystem choices (e.g. \texttt{typer, rich} as cli interface).
Proportionally speaking, the custom dataloader(s) and the main module have the highest \gls{LOC} counts.

\section{Language Models Considered}\label{sec:models}
\todo{figure out basic parameters (e.g. release date) of models used}
\todo{for each model, at the end, quickly put in one sentence why it was used / not used}
In this Section, we introduce the models used or considered for the benchmark.

See \secref{basics} for an overview and broad categorization of various models mentioned here.

\subsection{Criteria}\label{sub:criteria}
Setting out for our work, the only real constraint we had is that the model has to be any decently capable open-source \gls{causal}.

\subsection{OPT, BLOOM (Not Used)}
\paragraph{OPT}\label{par:opt}\label{sub:OPT}
The initial model set out for this work was \model{OPT} \cite{zhang_opt_2022}, a 175 billion parameter open-source \gls{LLM} trained by \gls{meta}, with partially similar capability as \gls{GPT3}. During early literature research, we encountered the similar but slightly more capable \model{BLOOM}.

\paragraph{BLOOM}\label{par:bloom}\label{sub:BLOOM}
\model{BLOOM} \cite{workshop_bloom_2022} is a 176 billion parameter open-source \gls{LLM} trained by a cooperation of numerous organizations, spearheaded by \gls{hf} and \gls{Google}. When compared to \model{OPT} across \gls{NLP} benchmarks, \model{BLOOM} appears to perform marginally better.

\paragraph{Reasons for using neither}
The original plan for this work would use \model{OPT} as only model. During early literature research, it seemed that \model{BLOOM} would be slightly more capable, so we intended to try it with both, and compare them. 
While still in literature research, \model{llama} got released, and with being seemingly both smaller and significantly more capable, as well as having properly scaled-down versions readily available, it was easy to make the call of going forward with \model{llama} only. 
\todo{slightly rewrite to more academic writing style}
See \subref{llama} for more details on \model{llama}.

\subsection{LLaMa (Used)}\label{sub:llama}
\model{llama} is a suite of open-access \glspl{LLM} from \gls{meta} with sizes ranging from 7 billion to 65 billion parameters, and capabilities comparable to, and sometimes beating \gls{SOTA} (including \gls{GPT3}) at the time \cite{touvron_llama_2023}. \model{llama} can be seen as the culmination of distributed progress in one place.

\model{llama} is not instruction fine-tuned. See \subref{instruct} for more details on instruction fine-tuning.
For instruction-finetuned variants of \model{llama}, see \subref{alpaca} on \model{alpaca} or \subref{vicuna} on \model{vicuna}.

\subsection{Alpaca (Not Used)}\label{sub:alpaca}
The \model{alpaca} Project \cite{tatsulab_2023} aims to build and share an instruction-finetuned \model{llama} model.
Due to uncertainty with the \model{llama} licence which this model is based on, no model weights where released officially.
They did, however, release everything else to easily fine-tune your own \model{alpaca} when you already have the weights for \model{llama}.
This becomes impractical for larger model variants due to increasing resource requirements. For this reason, we decided against including \model{alpaca} in our benchmark.

See \subref{instruct} for more details on instruction fine-tuning.

\subsection{Vicuna (Used Partially)}\label{sub:vicuna}
\model{vicuna} is a family of instruction fine-tuned \model{llama}-variants, released by \gls{lmsys}. It is building on top of the training recipe of \model{alpaca}.
However, not all weights of the corresponding \model{llama} sizes are available.
The largest \model{llama}-model (65B) does not have a corresponding \model{vicuna} derivative available.

in chatbot arena: beating out \model{llama} and \model{alpaca} \cite{zheng_judging_2023}
\todo{write out in more detail}

See \subref{instruct} for more details on instruction fine-tuning.

\subsection{LLaMa 2 (Used)}\label{sub:llama2}
\gls{meta} released \model{llama2} a few months after \model{llama}, in which they introduced few fundamental changes (making use of \gls{GQA} for the first time), trained on more tokens and released it under a different license. They also directly released its instruct variants.

See \subref{instruct} for more details on instruction fine-tuning.

\subsection{Falcon (Used)}\label{sub:falcon}
The \model{falcon} \cite{zxhang_falcon_2023} family of language models are created by the Abu Dhabi-based \gls{tii}.
\model{falcon} continues to dominate benchmarks with open-access models (in each respective parameter weight class), and also appears to rival some of the most capable closed-access models such as \gls{PaLM}.

Their better performance for most tasks is assumed to mostly the result of longer training and higher-quality data sets \cite{zxhang_falcon_2023}.

See \subref{instruct} for more details on instruction fine-tuning.

\subsection{GPT4 (Not Used)}\label{sub:GPT4}
\model{GPT4} is the fourth generation \gls{GPT} model from \gls{OpenAI} \cite{openai_gpt4_2023}.
It is the single most capable \acrlong{LM} we currently know of.
However, it is not open-source and only accessible through an API provided by \gls{OpenAI}.
Additionally, \gls{OpenAI} continues to work on, change, and measurably degrade the capabalities \cite{chen_how_2023} of \model{GPT4}, which makes it a bad target for comparison.
Even timestamped, supposedly 'unchanging' models have been claimed to measurably change in behaviour \cite{jw1224_hn}.

\subsection{Final List}\label{sub:list}
In conclusion, we used the following models and sizes of the aforementioned:
\begin{itemize}
    \item \model{llama} 7B, 13B, 30B, 65B (See \subref{llama} for more details on the model)
    \item \model{vicuna} 7B, 13B, 33B (See \subref{vicuna} for more details on the model)
    \item \model{llama2} 7B, 13B, 70B (See \subref{llama2} for more details on the model)
    \item \model{falcon} 7B, 40B (See \subref{falcon} for more details on the model)
    \item \model{falcon}-instruct 7B, 40B (See \subref{falcon} for more details on the model)
\end{itemize}


\section{Prompts Used}\label{sec:prompts}
\glspl{LLM} are capable of very generic tasks, based on the input they are asked to repsond to.
The way to get them to solve a task as requested is by \textit{prompting} the model with a certain input.
This is particularly emintent in instruct-based models (See \subref{instruct} for more details on instruction-based finetuning).

We have not put much effort in figuring out the best prompts, primarily because any amount of fine-tuning would be more effective than doing so. You can read more on what went wrong trying to do that in the following \secref{sft}.

As used \textit{guidance} \cite{guidance_2023}, and specifically the library \texttt{jsonformer} \cite{1rgs_2023}, for getting structured information as an output.
In effect, guidance provides `guard rails' for models generating output.
Specifically, the model does not have to generate the tokens for the structure of json, but only the tokens for the content of the json.

\code{schema.py}{schema}{The schema provided for the model to follow. Model output termination would happen after generation of a token for `\mintinline{python}{"}' for strings or `\texttt{,}' for numbers, or a number of other dedicated 'end of generation' tokens. See \coderef{output} for what an output for this schema might look like.}

You can see the schema we used for guidance in \coderef{schema}. Additionally, you can find the full prompt used in \coderef{prompt} and an example output in \coderef{output}.

\code{prompt.py}{prompt}{Prompt used to generate output. \mintinline{python}{"{output}"} delineates where the model provides an answer. See \coderef{output} for what may be filled in.}

\code{example_output.py}{output}{Exemplary output based on the prompt shown in \coderef{prompt}.}

% \subsection{Prompt Engineering}\label{sub:engineering}
% Answers, even to the same prompts, across models and even from the same model, can vary substantially \cite{chen_how_2023}.
% Thus, a short-lived 'discipline', Prompt Engineering, emerged.
% Prompt Engineering attempted to find out how to write prompts to get the best results, out of either specific or all models.
% It was quickly found out that this is a task that can be automated with the help of a \gls{LLM} \cite{zhou_large_2022}.

% additional relevancy for applications where potentially hostile users can directly or indirectly prompt a model, and thus 'Prompt Injection Attacks' where born \cite{greshake_more_2023}.

% \subsection{Prompt Guidelines}\label{sub:guidelines}
% \todo{totally rewrite}
% A few general guidelines for prompts empirically emerged (mostly through people sharing results on twitter):
% \begin{itemize}
%     \item Guidance for everything structure-based \cite{guidance_2023}
%     \item Chain-Of-Thought for reasoning \cite{wei_chainofthought_2022}
%     \item Reflexion for even bigger models \cite{shinn_reflexion_2023}
% \end{itemize}

% \subsection{Prompts Used}\label{sub:prompts}



\section{Supervised Fine Tuning}\label{sec:sft}
\todo{rewrite section on SFT}
The name of \acrfull{GPT} came from the fact that it simply was a large pre-trained transformer model which could be fine-tuned for any specific task.
The benefit of using using a pretrained model is that it requires substantially less compute, and maybe more importantly, examples to train on to achieve good results on a task \cite{gaddipati_comparative_2020}.
Until \gls{GPT3} became so capable that, for most \gls{NLP} tasks, you don't need to fine-tune it at all.

See \secref{training} for more details on pre-training a \gls{LLM} and \subref{finetune} for details on finetunig.

What follows are excerpts of attempting to fine-tune a model, and attempts at understanding why it didn't work.

\subsection{Excerpt 1: Broken Models}\label{sub:brokenft}
\todo{write subsection on broken models with more details}
demonstrate that stuff just isn't documented, anywhere, and even the community doesn't know.
\todo{figure out actual structure}

nuances: tokenization of dataset prior to training. however, which part is doing what?

building custom dataset: array with dicts, with the three required keys \verb`input_ids`, \verb`attention_maska`, and \verb`labels`. curiously, neither is documented particularly well so we tried what is recommended in various tutorials and official sources (e.g. microsoft \cite{deepspeedexamples_2023}): 

put the \verb`token_ids` received from tokenization to both \verb`input_ids` and \verb`labels`.

This did result in a model with differing weights than it had before. This model however, was broken as it did not generate anything that was not an EOS-token.
this token is usually used as a stopping criterion during generation.
? resulted in broken model, probably learned that it's 'finished', only outputting EOS tokens. Not sure if doing this otherwise would actually change anything though

Attempts at mask manipulation: not possible with causalLMs (they are all of this type)

\subsection{Excerpt 2: Broken Libraries}\label{sub:libraries}
In a later attempt wie tried using the high-level \gls{hf} \verb`trl` (Transformer Reinforcement Learning) library, which seems to be built for our use-case exactly.

However, this library is at best research-grade. The examples, while working with only a few lines, obscure the inner workings of the library.
And good luck: it's also not documented. There is the \verb`DataCollatorForCompletionOnlyLM` collator, which takes a tokenizer, but also doesn't tokenize?!?
\todo{rewrite subsection on broken libraries}

examples only have \verb`text` field, there is a formatting function and whatnot, but this implies tokenization is happening later. nope, errors with 'missing field \verb`token_ids`'.

trying out various things didn't work, until we ultimately didn't have time to continue.

SFT: a lot of magic that isn't documented properly, at all. Couldn't get it to run, gave up due to time limit.

% \mintinline{python}{trl} library \cite{hf_trl_supervised}

\section{Criteria for Equality}\label{sec:equality}
In this section we try to list the criteria we used to define equality between a result from a \gls{LLM} and the target label.
\todo{write section on criteria for equality}

for temperature and time we did conversion between units (not super straightforward), models had a bit of unit confusion
(sometimes adding too many or too few zeros, though also often getting it right)

solvents and additives: getting cid and comparing it (if it can be gotten in the first place though)

